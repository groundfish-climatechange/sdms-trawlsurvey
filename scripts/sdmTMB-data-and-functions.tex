% Options for packages loaded elsewhere
\PassOptionsToPackage{unicode=true}{hyperref}
\PassOptionsToPackage{hyphens}{url}
%
\documentclass[
]{article}
\usepackage{lmodern}
\usepackage{amssymb,amsmath}
\usepackage{ifxetex,ifluatex}
\ifnum 0\ifxetex 1\fi\ifluatex 1\fi=0 % if pdftex
  \usepackage[T1]{fontenc}
  \usepackage[utf8]{inputenc}
  \usepackage{textcomp} % provides euro and other symbols
\else % if luatex or xelatex
  \usepackage{unicode-math}
  \defaultfontfeatures{Scale=MatchLowercase}
  \defaultfontfeatures[\rmfamily]{Ligatures=TeX,Scale=1}
\fi
% Use upquote if available, for straight quotes in verbatim environments
\IfFileExists{upquote.sty}{\usepackage{upquote}}{}
\IfFileExists{microtype.sty}{% use microtype if available
  \usepackage[]{microtype}
  \UseMicrotypeSet[protrusion]{basicmath} % disable protrusion for tt fonts
}{}
\makeatletter
\@ifundefined{KOMAClassName}{% if non-KOMA class
  \IfFileExists{parskip.sty}{%
    \usepackage{parskip}
  }{% else
    \setlength{\parindent}{0pt}
    \setlength{\parskip}{6pt plus 2pt minus 1pt}}
}{% if KOMA class
  \KOMAoptions{parskip=half}}
\makeatother
\usepackage{xcolor}
\IfFileExists{xurl.sty}{\usepackage{xurl}}{} % add URL line breaks if available
\IfFileExists{bookmark.sty}{\usepackage{bookmark}}{\usepackage{hyperref}}
\hypersetup{
  pdftitle={Run sdmTMB},
  pdfauthor={Owen Liu},
  hidelinks,
}
\urlstyle{same} % disable monospaced font for URLs
\usepackage[margin=1in]{geometry}
\usepackage{color}
\usepackage{fancyvrb}
\newcommand{\VerbBar}{|}
\newcommand{\VERB}{\Verb[commandchars=\\\{\}]}
\DefineVerbatimEnvironment{Highlighting}{Verbatim}{commandchars=\\\{\}}
% Add ',fontsize=\small' for more characters per line
\usepackage{framed}
\definecolor{shadecolor}{RGB}{248,248,248}
\newenvironment{Shaded}{\begin{snugshade}}{\end{snugshade}}
\newcommand{\AlertTok}[1]{\textcolor[rgb]{0.94,0.16,0.16}{#1}}
\newcommand{\AnnotationTok}[1]{\textcolor[rgb]{0.56,0.35,0.01}{\textbf{\textit{#1}}}}
\newcommand{\AttributeTok}[1]{\textcolor[rgb]{0.77,0.63,0.00}{#1}}
\newcommand{\BaseNTok}[1]{\textcolor[rgb]{0.00,0.00,0.81}{#1}}
\newcommand{\BuiltInTok}[1]{#1}
\newcommand{\CharTok}[1]{\textcolor[rgb]{0.31,0.60,0.02}{#1}}
\newcommand{\CommentTok}[1]{\textcolor[rgb]{0.56,0.35,0.01}{\textit{#1}}}
\newcommand{\CommentVarTok}[1]{\textcolor[rgb]{0.56,0.35,0.01}{\textbf{\textit{#1}}}}
\newcommand{\ConstantTok}[1]{\textcolor[rgb]{0.00,0.00,0.00}{#1}}
\newcommand{\ControlFlowTok}[1]{\textcolor[rgb]{0.13,0.29,0.53}{\textbf{#1}}}
\newcommand{\DataTypeTok}[1]{\textcolor[rgb]{0.13,0.29,0.53}{#1}}
\newcommand{\DecValTok}[1]{\textcolor[rgb]{0.00,0.00,0.81}{#1}}
\newcommand{\DocumentationTok}[1]{\textcolor[rgb]{0.56,0.35,0.01}{\textbf{\textit{#1}}}}
\newcommand{\ErrorTok}[1]{\textcolor[rgb]{0.64,0.00,0.00}{\textbf{#1}}}
\newcommand{\ExtensionTok}[1]{#1}
\newcommand{\FloatTok}[1]{\textcolor[rgb]{0.00,0.00,0.81}{#1}}
\newcommand{\FunctionTok}[1]{\textcolor[rgb]{0.00,0.00,0.00}{#1}}
\newcommand{\ImportTok}[1]{#1}
\newcommand{\InformationTok}[1]{\textcolor[rgb]{0.56,0.35,0.01}{\textbf{\textit{#1}}}}
\newcommand{\KeywordTok}[1]{\textcolor[rgb]{0.13,0.29,0.53}{\textbf{#1}}}
\newcommand{\NormalTok}[1]{#1}
\newcommand{\OperatorTok}[1]{\textcolor[rgb]{0.81,0.36,0.00}{\textbf{#1}}}
\newcommand{\OtherTok}[1]{\textcolor[rgb]{0.56,0.35,0.01}{#1}}
\newcommand{\PreprocessorTok}[1]{\textcolor[rgb]{0.56,0.35,0.01}{\textit{#1}}}
\newcommand{\RegionMarkerTok}[1]{#1}
\newcommand{\SpecialCharTok}[1]{\textcolor[rgb]{0.00,0.00,0.00}{#1}}
\newcommand{\SpecialStringTok}[1]{\textcolor[rgb]{0.31,0.60,0.02}{#1}}
\newcommand{\StringTok}[1]{\textcolor[rgb]{0.31,0.60,0.02}{#1}}
\newcommand{\VariableTok}[1]{\textcolor[rgb]{0.00,0.00,0.00}{#1}}
\newcommand{\VerbatimStringTok}[1]{\textcolor[rgb]{0.31,0.60,0.02}{#1}}
\newcommand{\WarningTok}[1]{\textcolor[rgb]{0.56,0.35,0.01}{\textbf{\textit{#1}}}}
\usepackage{graphicx,grffile}
\makeatletter
\def\maxwidth{\ifdim\Gin@nat@width>\linewidth\linewidth\else\Gin@nat@width\fi}
\def\maxheight{\ifdim\Gin@nat@height>\textheight\textheight\else\Gin@nat@height\fi}
\makeatother
% Scale images if necessary, so that they will not overflow the page
% margins by default, and it is still possible to overwrite the defaults
% using explicit options in \includegraphics[width, height, ...]{}
\setkeys{Gin}{width=\maxwidth,height=\maxheight,keepaspectratio}
\setlength{\emergencystretch}{3em} % prevent overfull lines
\providecommand{\tightlist}{%
  \setlength{\itemsep}{0pt}\setlength{\parskip}{0pt}}
\setcounter{secnumdepth}{-\maxdimen} % remove section numbering
% Redefines (sub)paragraphs to behave more like sections
\ifx\paragraph\undefined\else
  \let\oldparagraph\paragraph
  \renewcommand{\paragraph}[1]{\oldparagraph{#1}\mbox{}}
\fi
\ifx\subparagraph\undefined\else
  \let\oldsubparagraph\subparagraph
  \renewcommand{\subparagraph}[1]{\oldsubparagraph{#1}\mbox{}}
\fi

% Set default figure placement to htbp
\makeatletter
\def\fps@figure{htbp}
\makeatother


\title{Run sdmTMB}
\author{Owen Liu}
\date{6/8/2021}

\begin{document}
\maketitle

{
\setcounter{tocdepth}{2}
\tableofcontents
}
\hypertarget{purpose}{%
\section{Purpose}\label{purpose}}

Join trawl survey data with ROMS oceanographic data and substrate data,
and write an sdmTMB wrapper to run models. Actual modelling done in
script \texttt{sdmTMB\ models.Rmd}

\hypertarget{import-data}{%
\section{Import Data}\label{import-data}}

\hypertarget{trawl-data}{%
\subsection{Trawl Data}\label{trawl-data}}

Trawl data, including a matching key to link to ROMS data. The hindcast
ROMS data has values for all trawl survey locations for all times, but
we just want the values matched to the actual trawl survey times/dates.

\begin{Shaded}
\begin{Highlighting}[]
\NormalTok{trawl <-}\StringTok{ }\KeywordTok{read_rds}\NormalTok{(}\KeywordTok{here}\NormalTok{(}\StringTok{'data'}\NormalTok{,}\StringTok{'nwfsc_trawl_data.rds'}\NormalTok{)) }\OperatorTok\StringTok{ }
\StringTok{  }\CommentTok{# convert date from character to date}
\StringTok{  }\KeywordTok{mutate}\NormalTok{(}\DataTypeTok{date=}\KeywordTok{as_date}\NormalTok{(date))}
\CommentTok{# roms time refernce}
\NormalTok{roms_time <-}\StringTok{ }\KeywordTok{read_rds}\NormalTok{(}\KeywordTok{here}\NormalTok{(}\StringTok{'data'}\NormalTok{,}\StringTok{'roms_time_date_reference.rds'}\NormalTok{))}
\NormalTok{trawl_locs <-}\StringTok{ }\KeywordTok{read_rds}\NormalTok{(here}\OperatorTok{::}\KeywordTok{here}\NormalTok{(}\StringTok{'data'}\NormalTok{,}\StringTok{'trawl'}\NormalTok{,}\StringTok{'trawlID.rds'}\NormalTok{)) }\OperatorTok\StringTok{ }
\StringTok{  }\CommentTok{# add a dummy indicator of a "real" trawl survey location in time/space}
\StringTok{  }\KeywordTok{mutate}\NormalTok{(}\DataTypeTok{date=}\KeywordTok{as_datetime}\NormalTok{(trawl_time,}\DataTypeTok{origin=}\StringTok{"1900-01-01"}\NormalTok{)) }\OperatorTok
\StringTok{  }\KeywordTok{mutate}\NormalTok{(}\DataTypeTok{date=}\KeywordTok{as_date}\NormalTok{(date)) }\OperatorTok
\StringTok{  }\KeywordTok{mutate}\NormalTok{(}\DataTypeTok{realTrawl=}\DecValTok{1}\NormalTok{) }\OperatorTok\StringTok{ }
\StringTok{  }\CommentTok{# join the roms_time reference}
\StringTok{  }\KeywordTok{left_join}\NormalTok{(roms_time) }\OperatorTok\StringTok{ }
\StringTok{  }\CommentTok{# select the variable we'll use to match}
\StringTok{  }\NormalTok{dplyr}\OperatorTok{::}\KeywordTok{select}\NormalTok{(station,date,time,lon_trawl,lat_trawl,depth_trawl,realTrawl)}
\end{Highlighting}
\end{Shaded}

\begin{verbatim}
## Joining, by = "date"
\end{verbatim}

\begin{Shaded}
\begin{Highlighting}[]
\NormalTok{trawl <-}\StringTok{ }\NormalTok{trawl }\OperatorTok\StringTok{ }
\StringTok{  }\KeywordTok{left_join}\NormalTok{(trawl_locs,}\DataTypeTok{by=}\KeywordTok{c}\NormalTok{(}\StringTok{"date"}\NormalTok{=}\StringTok{"date"}\NormalTok{,}\StringTok{"longitude_dd"}\NormalTok{=}\StringTok{"lon_trawl"}\NormalTok{,}\StringTok{"latitude_dd"}\NormalTok{=}\StringTok{"lat_trawl"}\NormalTok{,}\StringTok{"depth"}\NormalTok{=}\StringTok{"depth_trawl"}\NormalTok{))}
\end{Highlighting}
\end{Shaded}

Filter the trawl survey data to fit within the time frame for which we
have ROMS hindcast data (1980-2010)

\begin{Shaded}
\begin{Highlighting}[]
\NormalTok{trawl <-}\StringTok{ }\NormalTok{trawl }\OperatorTok
\StringTok{  }\KeywordTok{filter}\NormalTok{(realTrawl}\OperatorTok{==}\DecValTok{1}\NormalTok{) }\OperatorTok\StringTok{ }
\StringTok{  }\NormalTok{dplyr}\OperatorTok{::}\KeywordTok{select}\NormalTok{(}\OperatorTok{-}\NormalTok{realTrawl) }\OperatorTok\StringTok{ }
\StringTok{  }\CommentTok{# rename time to something more useful}
\StringTok{  }\KeywordTok{rename}\NormalTok{(}\DataTypeTok{roms_hindcast_day=}\NormalTok{time)}
\KeywordTok{glimpse}\NormalTok{(trawl)}
\end{Highlighting}
\end{Shaded}

\begin{verbatim}
## Rows: 374,688
## Columns: 25
## $ trawl_id            <dbl> 2.00303e+11, 2.00303e+11, 2.00303e+11, 2.00403e+11, 2.00503e+11, 2.00503e+11, ~
## $ scientific_name     <chr> "sebastes elongatus", "sebastes elongatus", "sebastes elongatus", "sebastes el~
## $ project             <chr> "NWFSC.Combo", "NWFSC.Combo", "NWFSC.Combo", "NWFSC.Combo", "NWFSC.Combo", "NW~
## $ year                <int> 2003, 2003, 2003, 2004, 2005, 2005, 2005, 2005, 2005, 2005, 2006, 2006, 2007, ~
## $ pass                <int> 1, 2, 1, 2, 2, 1, 1, 1, 1, 1, 1, 2, 1, 2, 1, 2, 2, 2, 2, 2, 2, 2, 2, 2, 2, 2, ~
## $ vessel              <chr> "Captain Jack", "Excalibur", "Ms. Julie", "Excalibur", "Excalibur", "Ms. Julie~
## $ tow                 <int> 111, 92, 119, 67, 205, 13, 38, 65, 116, 162, 169, 48, 154, 123, 169, 1, 2, 3, ~
## $ date                <date> 2003-07-29, 2003-09-27, 2003-07-30, 2004-09-06, 2005-10-16, 2005-05-31, 2005-~
## $ longitude_dd        <dbl> -123.1244, -124.6703, -122.5511, -124.4600, -118.6017, -124.2775, -124.4697, -~
## $ latitude_dd         <dbl> 38.07694, 41.48583, 37.37694, 43.67444, 33.65722, 45.83417, 46.31750, 43.85528~
## $ area_swept_ha       <dbl> 1.837650, 3.047135, 1.651004, 1.821803, 1.178600, 1.530574, 1.453994, 1.582128~
## $ subsample_count     <dbl> 0, 0, 0, 48, 0, 1, 6, 0, 16, 0, 1, 53, 0, 0, 0, 0, 0, 0, 0, 49, 0, 0, 2, 0, 0,~
## $ subsample_wt_kg     <dbl> 0.00, 0.00, 0.00, 10.30, 0.00, 0.12, 0.80, 0.00, 1.95, 0.00, 0.02, 14.10, 0.00~
## $ total_catch_numbers <dbl> 0, 0, 0, 48, 0, 1, 6, 0, 16, 0, 1, 242, 0, 0, 0, 0, 0, 0, 0, 81, 0, 0, 2, 0, 0~
## $ total_catch_wt_kg   <dbl> 0.00, 0.00, 0.00, 10.30, 0.00, 0.12, 0.80, 0.00, 1.95, 0.00, 0.02, 64.45, 0.00~
## $ cpue_kg_km2         <dbl> 0.000000, 0.000000, 0.000000, 565.373957, 0.000000, 7.840197, 55.020868, 0.000~
## $ species             <chr> "greenstriped rockfish", "greenstriped rockfish", "greenstriped rockfish", "gr~
## $ o2                  <dbl> NA, NA, NA, NA, NA, NA, NA, NA, NA, NA, NA, NA, NA, NA, NA, NA, NA, NA, NA, NA~
## $ temp                <dbl> 9.0359, 3.6300, 10.0280, 8.1160, 5.9237, 6.9375, 6.2750, 6.5258, 9.1859, 4.073~
## $ sal                 <dbl> NA, NA, NA, NA, NA, NA, NA, NA, NA, NA, NA, NA, NA, NA, NA, NA, NA, NA, NA, NA~
## $ depth               <dbl> 79.9095, 948.4094, 63.6213, 131.7826, 581.3504, 136.0367, 135.9776, 304.9347, ~
## $ performance         <chr> "Satisfactory", "Satisfactory", "Satisfactory", "Satisfactory", "Satisfactory"~
## $ survey              <chr> "nwfsc.combo", "nwfsc.combo", "nwfsc.combo", "nwfsc.combo", "nwfsc.combo", "nw~
## $ station             <int> 1, 2, 3, 4, 5, 6, 7, 8, 9, 10, 11, 12, 13, 14, 15, 16, 17, 18, 19, 20, 21, 22,~
## $ roms_hindcast_day   <int> 8611, 8671, 8612, 9016, 9421, 9283, 9289, 9302, 9321, 9336, 9695, 9740, 10053,~
\end{verbatim}

\hypertarget{roms-data}{%
\subsection{ROMS data}\label{roms-data}}

Hindcast ROMS data matched to trawl survey locations and times

\begin{Shaded}
\begin{Highlighting}[]
\NormalTok{roms <-}\StringTok{ }\KeywordTok{read_rds}\NormalTok{(here}\OperatorTok{::}\KeywordTok{here}\NormalTok{(}\StringTok{'data'}\NormalTok{,}\StringTok{'joined_30d_lagged_t_o.rds'}\NormalTok{)) }\OperatorTok
\StringTok{  }\CommentTok{# join the trawl_locs and filter by actual trawl locations and times}
\StringTok{  }\KeywordTok{left_join}\NormalTok{(trawl_locs) }\OperatorTok\StringTok{ }
\StringTok{  }\KeywordTok{filter}\NormalTok{(realTrawl}\OperatorTok{==}\DecValTok{1}\NormalTok{) }\OperatorTok\StringTok{ }
\StringTok{  }\KeywordTok{select}\NormalTok{(}\OperatorTok{-}\NormalTok{realTrawl) }\OperatorTok\StringTok{ }
\StringTok{  }\CommentTok{# rename time to something more useful}
\StringTok{  }\KeywordTok{rename}\NormalTok{(}\DataTypeTok{roms_hindcast_day=}\NormalTok{time)}
\end{Highlighting}
\end{Shaded}

\begin{verbatim}
## Joining, by = c("station", "time", "lon_trawl", "lat_trawl", "depth_trawl")
\end{verbatim}

\begin{Shaded}
\begin{Highlighting}[]
\KeywordTok{glimpse}\NormalTok{(roms)}
\end{Highlighting}
\end{Shaded}

\begin{verbatim}
## Rows: 4,832
## Columns: 13
## $ temp_roms           <dbl> 4.184411, 5.565145, 6.866655, 3.951781, 7.336621, 7.267605, 3.350144, 6.280363~
## $ oxygen_roms         <dbl> 44.78112, 78.12999, 137.91451, 41.38925, 168.28668, 166.14552, 28.76366, 123.9~
## $ station             <int> 274, 4671, 4772, 4773, 4845, 120, 4672, 4774, 4775, 4783, 4825, 275, 4840, 484~
## $ roms_hindcast_day   <int> 8576, 8576, 8576, 8576, 8576, 8577, 8577, 8577, 8577, 8577, 8577, 8578, 8578, ~
## $ trawl_time          <dbl> 3265444800, 3265444800, 3265444800, 3265444800, 3265444800, 3265531200, 326553~
## $ lon_trawl           <dbl> -124.7761, -124.7389, -124.5156, -124.7325, -124.5428, -124.8156, -125.1642, -~
## $ lat_trawl           <dbl> 46.09611, 46.02472, 46.15667, 46.50389, 46.75500, 47.60194, 46.81083, 47.67139~
## $ depth_trawl         <dbl> 564.9317, 310.0056, 140.7280, 606.3237, 107.4615, 106.0853, 797.7154, 176.6774~
## $ temp_trawl          <dbl> 4.8597, 5.6246, 6.7312, 4.6011, 6.8820, 6.9012, 3.8907, 6.7184, 6.7229, 6.6450~
## $ oxygen_trawl        <dbl> NaN, NaN, NaN, NaN, NaN, NaN, NaN, NaN, NaN, NaN, NaN, NaN, NaN, NaN, NaN, NaN~
## $ mean_temp_roms_30   <dbl> 4.227865, 5.564605, 6.970570, 4.016127, 7.363900, 7.492323, 3.361807, 6.455107~
## $ mean_oxygen_roms_30 <dbl> 45.80141, 81.00598, 141.60814, 42.14824, 170.73278, 178.97508, 28.82358, 128.2~
## $ date                <date> 2003-06-24, 2003-06-24, 2003-06-24, 2003-06-24, 2003-06-24, 2003-06-25, 2003-~
\end{verbatim}

\hypertarget{substrate-data}{%
\subsection{Substrate Data}\label{substrate-data}}

Here is the substrate data that Blake Feist matched to individual trawl
tows.

\begin{Shaded}
\begin{Highlighting}[]
\NormalTok{substrate <-}\StringTok{ }\KeywordTok{read_rds}\NormalTok{(}\KeywordTok{here}\NormalTok{(}\StringTok{'data'}\NormalTok{,}\StringTok{'substrate'}\NormalTok{,}\StringTok{'substrate_by_trawlID.rds'}\NormalTok{))}
\KeywordTok{glimpse}\NormalTok{(substrate)}
\end{Highlighting}
\end{Shaded}

\begin{verbatim}
## Rows: 20,746
## Columns: 11
## $ TRAWL_ID                                           <dbl> 1.97706e+11, 1.97706e+11, 1.97706e+11, 1.97706e~
## $ `Length_of_towline_outside_substrate_domain_(m)`   <dbl> 0, 0, 0, 0, 0, 0, 0, 0, 0, 0, 0, 0, 0, 0, 0, 0,~
## $ `Length_of_towline_traversing_hard_substrate_(m)`  <dbl> 0.00, 0.00, 0.00, 0.00, 0.00, 0.00, 0.00, 0.00,~
## $ `Length_of_towline_traversing_mixed_substrate_(m)` <dbl> 0.00, 0.00, 169.37, 0.00, 0.00, 0.00, 0.00, 0.0~
## $ `Length_of_towline_traversing_soft_substrate_(m)`  <dbl> 2628.42, 2484.34, 2899.37, 2570.00, 2639.27, 25~
## $ `Total_length_of_towline_(m)`                      <dbl> 2628.42, 2484.34, 3068.74, 2570.00, 2639.27, 25~
## $ Proportion_outside_substrate_domain                <dbl> 0, 0, 0, 0, 0, 0, 0, 0, 0, 0, 0, 0, 0, 0, 0, 0,~
## $ Proportion_hard                                    <dbl> 0.0000000, 0.0000000, 0.0000000, 0.0000000, 0.0~
## $ Proportion_mixed                                   <dbl> 0.00000000, 0.00000000, 0.05519243, 0.00000000,~
## $ Proportion_soft                                    <dbl> 1.00000000, 1.00000000, 0.94480757, 1.00000000,~
## $ prop_hard_mixed                                    <dbl> 0.00000000, 0.00000000, 0.05519243, 0.00000000,~
\end{verbatim}

Look at the form of these data

\begin{Shaded}
\begin{Highlighting}[]
\KeywordTok{glimpse}\NormalTok{(trawl)}
\end{Highlighting}
\end{Shaded}

\begin{verbatim}
## Rows: 374,688
## Columns: 25
## $ trawl_id            <dbl> 2.00303e+11, 2.00303e+11, 2.00303e+11, 2.00403e+11, 2.00503e+11, 2.00503e+11, ~
## $ scientific_name     <chr> "sebastes elongatus", "sebastes elongatus", "sebastes elongatus", "sebastes el~
## $ project             <chr> "NWFSC.Combo", "NWFSC.Combo", "NWFSC.Combo", "NWFSC.Combo", "NWFSC.Combo", "NW~
## $ year                <int> 2003, 2003, 2003, 2004, 2005, 2005, 2005, 2005, 2005, 2005, 2006, 2006, 2007, ~
## $ pass                <int> 1, 2, 1, 2, 2, 1, 1, 1, 1, 1, 1, 2, 1, 2, 1, 2, 2, 2, 2, 2, 2, 2, 2, 2, 2, 2, ~
## $ vessel              <chr> "Captain Jack", "Excalibur", "Ms. Julie", "Excalibur", "Excalibur", "Ms. Julie~
## $ tow                 <int> 111, 92, 119, 67, 205, 13, 38, 65, 116, 162, 169, 48, 154, 123, 169, 1, 2, 3, ~
## $ date                <date> 2003-07-29, 2003-09-27, 2003-07-30, 2004-09-06, 2005-10-16, 2005-05-31, 2005-~
## $ longitude_dd        <dbl> -123.1244, -124.6703, -122.5511, -124.4600, -118.6017, -124.2775, -124.4697, -~
## $ latitude_dd         <dbl> 38.07694, 41.48583, 37.37694, 43.67444, 33.65722, 45.83417, 46.31750, 43.85528~
## $ area_swept_ha       <dbl> 1.837650, 3.047135, 1.651004, 1.821803, 1.178600, 1.530574, 1.453994, 1.582128~
## $ subsample_count     <dbl> 0, 0, 0, 48, 0, 1, 6, 0, 16, 0, 1, 53, 0, 0, 0, 0, 0, 0, 0, 49, 0, 0, 2, 0, 0,~
## $ subsample_wt_kg     <dbl> 0.00, 0.00, 0.00, 10.30, 0.00, 0.12, 0.80, 0.00, 1.95, 0.00, 0.02, 14.10, 0.00~
## $ total_catch_numbers <dbl> 0, 0, 0, 48, 0, 1, 6, 0, 16, 0, 1, 242, 0, 0, 0, 0, 0, 0, 0, 81, 0, 0, 2, 0, 0~
## $ total_catch_wt_kg   <dbl> 0.00, 0.00, 0.00, 10.30, 0.00, 0.12, 0.80, 0.00, 1.95, 0.00, 0.02, 64.45, 0.00~
## $ cpue_kg_km2         <dbl> 0.000000, 0.000000, 0.000000, 565.373957, 0.000000, 7.840197, 55.020868, 0.000~
## $ species             <chr> "greenstriped rockfish", "greenstriped rockfish", "greenstriped rockfish", "gr~
## $ o2                  <dbl> NA, NA, NA, NA, NA, NA, NA, NA, NA, NA, NA, NA, NA, NA, NA, NA, NA, NA, NA, NA~
## $ temp                <dbl> 9.0359, 3.6300, 10.0280, 8.1160, 5.9237, 6.9375, 6.2750, 6.5258, 9.1859, 4.073~
## $ sal                 <dbl> NA, NA, NA, NA, NA, NA, NA, NA, NA, NA, NA, NA, NA, NA, NA, NA, NA, NA, NA, NA~
## $ depth               <dbl> 79.9095, 948.4094, 63.6213, 131.7826, 581.3504, 136.0367, 135.9776, 304.9347, ~
## $ performance         <chr> "Satisfactory", "Satisfactory", "Satisfactory", "Satisfactory", "Satisfactory"~
## $ survey              <chr> "nwfsc.combo", "nwfsc.combo", "nwfsc.combo", "nwfsc.combo", "nwfsc.combo", "nw~
## $ station             <int> 1, 2, 3, 4, 5, 6, 7, 8, 9, 10, 11, 12, 13, 14, 15, 16, 17, 18, 19, 20, 21, 22,~
## $ roms_hindcast_day   <int> 8611, 8671, 8612, 9016, 9421, 9283, 9289, 9302, 9321, 9336, 9695, 9740, 10053,~
\end{verbatim}

\begin{Shaded}
\begin{Highlighting}[]
\KeywordTok{glimpse}\NormalTok{(roms)}
\end{Highlighting}
\end{Shaded}

\begin{verbatim}
## Rows: 4,832
## Columns: 13
## $ temp_roms           <dbl> 4.184411, 5.565145, 6.866655, 3.951781, 7.336621, 7.267605, 3.350144, 6.280363~
## $ oxygen_roms         <dbl> 44.78112, 78.12999, 137.91451, 41.38925, 168.28668, 166.14552, 28.76366, 123.9~
## $ station             <int> 274, 4671, 4772, 4773, 4845, 120, 4672, 4774, 4775, 4783, 4825, 275, 4840, 484~
## $ roms_hindcast_day   <int> 8576, 8576, 8576, 8576, 8576, 8577, 8577, 8577, 8577, 8577, 8577, 8578, 8578, ~
## $ trawl_time          <dbl> 3265444800, 3265444800, 3265444800, 3265444800, 3265444800, 3265531200, 326553~
## $ lon_trawl           <dbl> -124.7761, -124.7389, -124.5156, -124.7325, -124.5428, -124.8156, -125.1642, -~
## $ lat_trawl           <dbl> 46.09611, 46.02472, 46.15667, 46.50389, 46.75500, 47.60194, 46.81083, 47.67139~
## $ depth_trawl         <dbl> 564.9317, 310.0056, 140.7280, 606.3237, 107.4615, 106.0853, 797.7154, 176.6774~
## $ temp_trawl          <dbl> 4.8597, 5.6246, 6.7312, 4.6011, 6.8820, 6.9012, 3.8907, 6.7184, 6.7229, 6.6450~
## $ oxygen_trawl        <dbl> NaN, NaN, NaN, NaN, NaN, NaN, NaN, NaN, NaN, NaN, NaN, NaN, NaN, NaN, NaN, NaN~
## $ mean_temp_roms_30   <dbl> 4.227865, 5.564605, 6.970570, 4.016127, 7.363900, 7.492323, 3.361807, 6.455107~
## $ mean_oxygen_roms_30 <dbl> 45.80141, 81.00598, 141.60814, 42.14824, 170.73278, 178.97508, 28.82358, 128.2~
## $ date                <date> 2003-06-24, 2003-06-24, 2003-06-24, 2003-06-24, 2003-06-24, 2003-06-25, 2003-~
\end{verbatim}

\begin{Shaded}
\begin{Highlighting}[]
\KeywordTok{glimpse}\NormalTok{(substrate)}
\end{Highlighting}
\end{Shaded}

\begin{verbatim}
## Rows: 20,746
## Columns: 11
## $ TRAWL_ID                                           <dbl> 1.97706e+11, 1.97706e+11, 1.97706e+11, 1.97706e~
## $ `Length_of_towline_outside_substrate_domain_(m)`   <dbl> 0, 0, 0, 0, 0, 0, 0, 0, 0, 0, 0, 0, 0, 0, 0, 0,~
## $ `Length_of_towline_traversing_hard_substrate_(m)`  <dbl> 0.00, 0.00, 0.00, 0.00, 0.00, 0.00, 0.00, 0.00,~
## $ `Length_of_towline_traversing_mixed_substrate_(m)` <dbl> 0.00, 0.00, 169.37, 0.00, 0.00, 0.00, 0.00, 0.0~
## $ `Length_of_towline_traversing_soft_substrate_(m)`  <dbl> 2628.42, 2484.34, 2899.37, 2570.00, 2639.27, 25~
## $ `Total_length_of_towline_(m)`                      <dbl> 2628.42, 2484.34, 3068.74, 2570.00, 2639.27, 25~
## $ Proportion_outside_substrate_domain                <dbl> 0, 0, 0, 0, 0, 0, 0, 0, 0, 0, 0, 0, 0, 0, 0, 0,~
## $ Proportion_hard                                    <dbl> 0.0000000, 0.0000000, 0.0000000, 0.0000000, 0.0~
## $ Proportion_mixed                                   <dbl> 0.00000000, 0.00000000, 0.05519243, 0.00000000,~
## $ Proportion_soft                                    <dbl> 1.00000000, 1.00000000, 0.94480757, 1.00000000,~
## $ prop_hard_mixed                                    <dbl> 0.00000000, 0.00000000, 0.05519243, 0.00000000,~
\end{verbatim}

For the ROMS data (for now), we are using modelled temperature and
oxygen, lagged 30 days from each trawl survey location and time.

\begin{Shaded}
\begin{Highlighting}[]
\NormalTok{roms_thin <-}\StringTok{ }\NormalTok{roms }\OperatorTok\StringTok{ }
\StringTok{  }\NormalTok{dplyr}\OperatorTok{::}\KeywordTok{select}\NormalTok{(station,lon_trawl,lat_trawl,depth_trawl,mean_temp_roms_}\DecValTok{30}\NormalTok{,mean_oxygen_roms_}\DecValTok{30}\NormalTok{)}
\end{Highlighting}
\end{Shaded}

\hypertarget{join-datasets}{%
\section{Join Datasets}\label{join-datasets}}

\hypertarget{join-trawl-and-roms}{%
\section{Join Trawl and ROMS}\label{join-trawl-and-roms}}

Join the two datasets together, such that we have the
appropriately-matched ROMS outputs

\begin{Shaded}
\begin{Highlighting}[]
\NormalTok{trawl_roms <-}\StringTok{ }\NormalTok{trawl }\OperatorTok\StringTok{ }\KeywordTok{left_join}\NormalTok{(roms,}\DataTypeTok{by =} \KeywordTok{c}\NormalTok{(}\StringTok{"date"}\NormalTok{, }\StringTok{"station"}\NormalTok{, }\StringTok{"roms_hindcast_day"}\NormalTok{)) }\OperatorTok\StringTok{ }
\StringTok{  }\CommentTok{#clean up some columns}
\StringTok{  }\NormalTok{dplyr}\OperatorTok{::}\KeywordTok{select}\NormalTok{(date,trawl_id,station,lon_trawl,lat_trawl,depth_trawl,mean_temp_roms_}\DecValTok{30}\NormalTok{,mean_oxygen_roms_}\DecValTok{30}\NormalTok{,species,cpue_kg_km2) }\OperatorTok\StringTok{ }
\StringTok{  }\CommentTok{# drop any rows with NAs}
\StringTok{  }\KeywordTok{drop_na}\NormalTok{()}
\end{Highlighting}
\end{Shaded}

\begin{Shaded}
\begin{Highlighting}[]
  \CommentTok{# test <- trawl %>% left_join(roms,by = c("date", "station", "roms_hindcast_day")) %>% }
  \CommentTok{#   #clean up some columns}
  \CommentTok{#   dplyr::select(date,trawl_id,station,lon_trawl,lat_trawl,depth_trawl,temp_roms,oxygen_roms,mean_temp_roms_30,mean_oxygen_roms_30,species,cpue_kg_km2) %>% }
  \CommentTok{#   # drop any rows with NAs}
  \CommentTok{#   drop_na()}
\end{Highlighting}
\end{Shaded}

\hypertarget{join-trawl-and-substrate}{%
\subsection{Join Trawl and Substrate}\label{join-trawl-and-substrate}}

Join the substrate data by trawl ID number.

\begin{Shaded}
\begin{Highlighting}[]
\NormalTok{substrate_thin <-}\StringTok{ }\NormalTok{substrate }\OperatorTok\StringTok{ }
\StringTok{  }\NormalTok{dplyr}\OperatorTok{::}\KeywordTok{select}\NormalTok{(TRAWL_ID,prop_hard_mixed)}
\NormalTok{trawl_roms <-}\StringTok{ }\NormalTok{trawl_roms }\OperatorTok\StringTok{ }
\StringTok{  }\KeywordTok{left_join}\NormalTok{(substrate_thin,}\DataTypeTok{by=}\KeywordTok{c}\NormalTok{(}\StringTok{'trawl_id'}\NormalTok{=}\StringTok{"TRAWL_ID"}\NormalTok{)) }\OperatorTok\StringTok{ }
\StringTok{  }\KeywordTok{drop_na}\NormalTok{()}
\end{Highlighting}
\end{Shaded}

\hypertarget{prepare-data-for-sdmtmb}{%
\section{Prepare Data for sdmTMB}\label{prepare-data-for-sdmtmb}}

Convert the trawl spatial data to UTM.

\begin{Shaded}
\begin{Highlighting}[]
\CommentTok{# convert to UTM}
\NormalTok{trawl_roms_utm <-}\StringTok{ }\NormalTok{trawl_roms }\OperatorTok
\StringTok{  }\CommentTok{# convert to sf object}
\StringTok{  }\KeywordTok{st_as_sf}\NormalTok{(}\DataTypeTok{coords=}\KeywordTok{c}\NormalTok{(}\StringTok{'lon_trawl'}\NormalTok{,}\StringTok{'lat_trawl'}\NormalTok{),}\DataTypeTok{crs=}\DecValTok{4326}\NormalTok{) }\OperatorTok
\StringTok{  }\CommentTok{# transform to UTM zone 10}
\StringTok{  }\KeywordTok{st_transform}\NormalTok{(}\DataTypeTok{crs =} \StringTok{"+proj=utm +zone=10 +datum=WGS84 +units=km"}\NormalTok{) }\OperatorTok
\StringTok{  }\CommentTok{# add new coords as vars}
\StringTok{  }\KeywordTok{mutate}\NormalTok{(}\DataTypeTok{latitude =}\NormalTok{ sf}\OperatorTok{::}\KeywordTok{st_coordinates}\NormalTok{(.)[,}\DecValTok{2}\NormalTok{],}
         \DataTypeTok{longitude =}\NormalTok{ sf}\OperatorTok{::}\KeywordTok{st_coordinates}\NormalTok{(.)[,}\DecValTok{1}\NormalTok{]) }\OperatorTok
\StringTok{  }\CommentTok{# convert back to normal df}
\StringTok{  }\KeywordTok{st_set_geometry}\NormalTok{(}\OtherTok{NULL}\NormalTok{)}
\end{Highlighting}
\end{Shaded}

We can save this version of the data so we do not have to run the join
every time.

\begin{Shaded}
\begin{Highlighting}[]
\KeywordTok{write_rds}\NormalTok{(trawl_roms_utm,here}\OperatorTok{::}\KeywordTok{here}\NormalTok{(}\StringTok{'data'}\NormalTok{,}\StringTok{'trawl_roms_joined.rds'}\NormalTok{))}
\end{Highlighting}
\end{Shaded}

\hypertarget{functions-to-run-a-model}{%
\section{Functions to Run a Model}\label{functions-to-run-a-model}}

We'll write two functions, one to prepare a specific species for an
sdmTMB model, and another to actually run an sdmTMB model with custom
options

\hypertarget{prepare-species-data}{%
\subsection{Prepare Species' Data}\label{prepare-species-data}}

This function selects a species' data from the trawl survey data,
converts the spatial data to UTM, does a couple of filters for missing
data, and then joins the ROMS hindcast data to it by time and location.

\begin{Shaded}
\begin{Highlighting}[]
\NormalTok{prepare_species <-}\StringTok{ }\ControlFlowTok{function}\NormalTok{(dat,spp)\{}
\NormalTok{  dat_sub <-}\StringTok{ }\NormalTok{dat }\OperatorTok\StringTok{ }
\StringTok{    }\KeywordTok{filter}\NormalTok{(species}\OperatorTok{==}\NormalTok{spp) }\OperatorTok\StringTok{ }
\StringTok{    }
\StringTok{    }\CommentTok{# rescale depth, oxygen, and temp to be N(0,1)}
\StringTok{    }\KeywordTok{mutate}\NormalTok{(}\KeywordTok{across}\NormalTok{(}\KeywordTok{c}\NormalTok{(depth_trawl,mean_temp_roms_}\DecValTok{30}\NormalTok{,mean_oxygen_roms_}\DecValTok{30}\NormalTok{),}\OperatorTok{~}\NormalTok{(}\KeywordTok{scale}\NormalTok{(.) }\OperatorTok\StringTok{ }\KeywordTok{as.vector}\NormalTok{()),}\DataTypeTok{.names=}\StringTok{"\{.col\}_norm"}\NormalTok{)) }\OperatorTok\StringTok{ }
\StringTok{    }
\StringTok{    }\CommentTok{# add a year indicator}
\StringTok{    }\KeywordTok{mutate}\NormalTok{(}\DataTypeTok{year=}\NormalTok{lubridate}\OperatorTok{::}\KeywordTok{year}\NormalTok{(date))}
\NormalTok{\}}
\end{Highlighting}
\end{Shaded}

Try an example for sablefish

\begin{Shaded}
\begin{Highlighting}[]
\NormalTok{sablefish_dat <-}\StringTok{ }\KeywordTok{prepare_species}\NormalTok{(trawl_roms_utm,}\DataTypeTok{spp=}\StringTok{"sablefish"}\NormalTok{)}
\KeywordTok{glimpse}\NormalTok{(sablefish_dat)}
\end{Highlighting}
\end{Shaded}

\hypertarget{sdmtmb-model-function}{%
\subsection{sdmTMB Model Function}\label{sdmtmb-model-function}}

Write a function that runs sdmTMB. It wil call the previous function to
make the appropriate species data. For now, the environmental variable
names are not generic (always \texttt{mean\_temp\_roms\_30\_norm} and
\texttt{mean\_oxygen\_roms\_30\_norm})

\begin{Shaded}
\begin{Highlighting}[]
\CommentTok{# nknots=400;use_depth=F;time_vary=F;spatial_field=T;hab_spline=F;env_spline=F;spline_k=3}
\CommentTok{# rm(time_varying,spatial_field,hab_spline,env_spline,spline_k,modeldat,spde,formula,substrate,enviro,dat,spp,time_formula,time,test_set,nknots,test_cv,use_depth,time_vary)}
\NormalTok{run_sdmTMB <-}\StringTok{ }\ControlFlowTok{function}\NormalTok{(dat,spp,}\DataTypeTok{nknots=}\DecValTok{400}\NormalTok{,}\DataTypeTok{use_depth=}\NormalTok{F,}\DataTypeTok{time_vary=}\NormalTok{F,}\DataTypeTok{spatial_field=}\NormalTok{T,}\DataTypeTok{hab_spline=}\NormalTok{F,}\DataTypeTok{env_spline=}\NormalTok{F,}\DataTypeTok{spline_k=}\DecValTok{3}\NormalTok{)\{}
  \CommentTok{# filter data for species}
\NormalTok{  modeldat <-}\StringTok{ }\KeywordTok{prepare_species}\NormalTok{(dat,}\DataTypeTok{spp=}\NormalTok{spp)}
  
  \CommentTok{# make spde}
\NormalTok{  spde <-}\StringTok{ }\KeywordTok{make_mesh}\NormalTok{(modeldat,}\DataTypeTok{xy_cols =} \KeywordTok{c}\NormalTok{(}\StringTok{'longitude'}\NormalTok{,}\StringTok{'latitude'}\NormalTok{), }
                   \DataTypeTok{cutoff =} \DecValTok{20}\NormalTok{)}
  
  \CommentTok{# model formula}
\NormalTok{  formula <-}\StringTok{ }\KeywordTok{paste0}\NormalTok{(}\StringTok{"cpue_kg_km2 ~ "}\NormalTok{)}
  
  \CommentTok{# substrate relationship}
\NormalTok{  substrate <-}\StringTok{ }\KeywordTok{paste}\NormalTok{(}\StringTok{"prop_hard_mixed + I(prop_hard_mixed^2)"}\NormalTok{)}
  \CommentTok{#wiggly habitat relationship?}
\NormalTok{  substrate <-}\StringTok{ }\KeywordTok{ifelse}\NormalTok{(hab_spline, }\KeywordTok{paste0}\NormalTok{(}\StringTok{"s(prop_hard_mixed,k="}\NormalTok{,spline_k,}\StringTok{")"}\NormalTok{),}
\NormalTok{                      substrate)}
  
  \CommentTok{# make the environmental effects}
\NormalTok{  enviro <-}\StringTok{ }\KeywordTok{paste}\NormalTok{(}\StringTok{"mean_temp_roms_30_norm + }
\StringTok{                  I(mean_temp_roms_30_norm^2) + }
\StringTok{                  mean_oxygen_roms_30_norm + }
\StringTok{                  I(mean_oxygen_roms_30_norm^2)"}\NormalTok{)}
  \CommentTok{# wiggly environmental relationships?}
\NormalTok{  enviro <-}\StringTok{ }\KeywordTok{ifelse}\NormalTok{(env_spline, }\KeywordTok{paste0}\NormalTok{(}\StringTok{"s(mean_temp_roms_30_norm,k="}\NormalTok{,spline_k,}\StringTok{") + "}\NormalTok{,}
                                      \StringTok{"s(mean_oxygen_roms_30_norm,k="}\NormalTok{,spline_k,}\StringTok{")"}\NormalTok{),}
\NormalTok{                   enviro)}
  \CommentTok{# if depth effect, add to model formla}
  \ControlFlowTok{if}\NormalTok{(use_depth) \{}
\NormalTok{    formula =}\StringTok{ }\KeywordTok{paste0}\NormalTok{(formula, }\StringTok{" + depth + I(depth^2)"}\NormalTok{)}
\NormalTok{  \}}
  
\NormalTok{  time_formula =}\StringTok{ "~ -1"}
  \ControlFlowTok{if}\NormalTok{(time_vary) \{}
\NormalTok{    time_formula =}\StringTok{ }\KeywordTok{paste0}\NormalTok{(time_formula, }\StringTok{" + "}\NormalTok{, substrate, }\StringTok{" + "}\NormalTok{, enviro)}
\NormalTok{    time_varying =}\StringTok{ }\KeywordTok{as.formula}\NormalTok{(time_formula)}
\NormalTok{    time =}\StringTok{ "year"}
\NormalTok{  \} }\ControlFlowTok{else}\NormalTok{ \{}
\NormalTok{    formula =}\StringTok{ }\KeywordTok{paste0}\NormalTok{(formula, }\StringTok{" + "}\NormalTok{, substrate, }\StringTok{" + "}\NormalTok{, enviro)}
\NormalTok{    time_varying =}\StringTok{ }\OtherTok{NULL}
\NormalTok{    time =}\StringTok{ "year"}
\NormalTok{  \}}
  
  \CommentTok{# fit model. EW commented out quadratic roots, since those are still experimental and won't work for all spp. Also turned}
  \CommentTok{# set.seed(41389) # for reproducibility}
  \CommentTok{# test_set = sample(1:nrow(modeldat), size = round(0.1*nrow(modeldat)), replace=FALSE)}
  \CommentTok{# modeldat$fold = 1}
  \CommentTok{# modeldat$fold[test_set] = 2}
  \CommentTok{# anisotropy off for now}
  \KeywordTok{print}\NormalTok{(}\StringTok{'running model.'}\NormalTok{)}
\NormalTok{  m <-}\StringTok{ }\KeywordTok{try}\NormalTok{( }\KeywordTok{sdmTMB}\NormalTok{(}
    \DataTypeTok{formula =} \KeywordTok{as.formula}\NormalTok{(formula),}
    \DataTypeTok{time_varying =}\NormalTok{ time_varying,}
    \DataTypeTok{spde =}\NormalTok{ spde,}
    \DataTypeTok{time =}\NormalTok{ time,}
    \DataTypeTok{family =} \KeywordTok{tweedie}\NormalTok{(}\DataTypeTok{link =} \StringTok{"log"}\NormalTok{),}
    \DataTypeTok{data =}\NormalTok{ modeldat,}
    \DataTypeTok{anisotropy =} \OtherTok{FALSE}\NormalTok{,}
    \DataTypeTok{spatial_only =}\NormalTok{ T,}
    \CommentTok{#extra_time argument necessary for prediction?}
    \DataTypeTok{extra_time=}\DecValTok{1980}\OperatorTok{:}\DecValTok{2100}\NormalTok{,}
    \DataTypeTok{control=}\KeywordTok{sdmTMBcontrol}\NormalTok{(}\DataTypeTok{map_rf=}\KeywordTok{ifelse}\NormalTok{(spatial_field,F,T))}
\NormalTok{  ),}
  \DataTypeTok{silent=}\NormalTok{F)}


  \CommentTok{# predicted values for the 2nd fold (test)}
  \CommentTok{# m_cv$data$cv_predicted[which(m_cv$data$cv_fold==2)]}
  \CommentTok{# log likelihood values for the 2nd fold (test)}
  \CommentTok{# m_cv$data$cv_loglik[which(m_cv$data$cv_fold==2)]}

    \CommentTok{# sum(m_cv$data$cv_loglik[which(m_cv$data$cv_fold==2)])}
  
  \CommentTok{# if(class(m)!="try-error") \{}
  \CommentTok{#   write_rds(m, file=here::here('model output',}
  \CommentTok{#                                paste0(spp,'.rds')))}
  \CommentTok{# \}}
  \ControlFlowTok{if}\NormalTok{(}\KeywordTok{class}\NormalTok{(m)}\OperatorTok{==}\StringTok{"try-error"}\NormalTok{)\{}
    \KeywordTok{print}\NormalTok{(}\KeywordTok{paste}\NormalTok{(}\StringTok{"Error."}\NormalTok{))}
\NormalTok{  \}}\ControlFlowTok{else}\NormalTok{\{}
    \KeywordTok{print}\NormalTok{(}\KeywordTok{paste}\NormalTok{(}\StringTok{"Model for"}\NormalTok{,spp,}\StringTok{"complete."}\NormalTok{))}
\NormalTok{  \}}

  \CommentTok{# return(m)}
  \KeywordTok{return}\NormalTok{(m)}

\NormalTok{\}}
\end{Highlighting}
\end{Shaded}

\begin{Shaded}
\begin{Highlighting}[]
\NormalTok{test <-}\StringTok{ }\KeywordTok{run_sdmTMB}\NormalTok{(}\DataTypeTok{dat=}\NormalTok{trawl_roms_utm,}\DataTypeTok{spp=}\StringTok{"sablefish"}\NormalTok{,}\DataTypeTok{hab_spline =}\NormalTok{ F,}\DataTypeTok{env_spline =}\NormalTok{ F)}
\end{Highlighting}
\end{Shaded}

\#\#CV formula

\begin{Shaded}
\begin{Highlighting}[]
\KeywordTok{library}\NormalTok{(future)}
\KeywordTok{plan}\NormalTok{(multisession)}
\NormalTok{run_sdmTMB_cv <-}\StringTok{ }\ControlFlowTok{function}\NormalTok{(dat,spp,}\DataTypeTok{nknots=}\DecValTok{400}\NormalTok{,}\DataTypeTok{use_depth=}\NormalTok{F,}\DataTypeTok{time_vary=}\NormalTok{F,}\DataTypeTok{spatial_field=}\NormalTok{T,}\DataTypeTok{hab_spline=}\NormalTok{F,}\DataTypeTok{env_spline=}\NormalTok{F,}\DataTypeTok{spline_k=}\DecValTok{3}\NormalTok{,}\DataTypeTok{return_what=}\StringTok{'loglik'}\NormalTok{)\{}
  \CommentTok{# filter data for species}
\NormalTok{  modeldat <-}\StringTok{ }\KeywordTok{prepare_species}\NormalTok{(dat,}\DataTypeTok{spp=}\NormalTok{spp)}
  
  \CommentTok{# make spde}
\NormalTok{  spde <-}\StringTok{ }\KeywordTok{make_mesh}\NormalTok{(modeldat,}\DataTypeTok{xy_cols =} \KeywordTok{c}\NormalTok{(}\StringTok{'longitude'}\NormalTok{,}\StringTok{'latitude'}\NormalTok{), }
                   \DataTypeTok{cutoff =} \DecValTok{20}\NormalTok{)}
  
  \CommentTok{# model formula}
\NormalTok{  formula <-}\StringTok{ }\KeywordTok{paste0}\NormalTok{(}\StringTok{"cpue_kg_km2 ~ "}\NormalTok{)}
  
  \CommentTok{# substrate relationship}
\NormalTok{  substrate <-}\StringTok{ }\KeywordTok{paste}\NormalTok{(}\StringTok{"prop_hard_mixed + I(prop_hard_mixed^2)"}\NormalTok{)}
  \CommentTok{#wiggly habitat relationship?}
\NormalTok{  substrate <-}\StringTok{ }\KeywordTok{ifelse}\NormalTok{(hab_spline, }\KeywordTok{paste0}\NormalTok{(}\StringTok{"s(prop_hard_mixed,k="}\NormalTok{,spline_k,}\StringTok{")"}\NormalTok{),}
\NormalTok{                      substrate)}
  
  \CommentTok{# make the environmental effects}
\NormalTok{  enviro <-}\StringTok{ }\KeywordTok{paste}\NormalTok{(}\StringTok{"mean_temp_roms_30_norm + I(mean_temp_roms_30_norm^2) + mean_oxygen_roms_30_norm + I(mean_oxygen_roms_30_norm^2)"}\NormalTok{)}
  \CommentTok{# wiggly environmental relationships?}
\NormalTok{  enviro <-}\StringTok{ }\KeywordTok{ifelse}\NormalTok{(env_spline, }\KeywordTok{paste0}\NormalTok{(}\StringTok{"s(mean_temp_roms_30_norm,k="}\NormalTok{,spline_k,}\StringTok{") + "}\NormalTok{,}
                                      \StringTok{"s(mean_oxygen_roms_30_norm,k="}\NormalTok{,spline_k,}\StringTok{")"}\NormalTok{),}
\NormalTok{                   enviro)}
  \CommentTok{# if depth effect, add to model formla}
  \ControlFlowTok{if}\NormalTok{(use_depth) \{}
\NormalTok{    formula =}\StringTok{ }\KeywordTok{paste0}\NormalTok{(formula, }\StringTok{" + depth + I(depth^2)"}\NormalTok{)}
\NormalTok{  \}}
  
\NormalTok{  time_formula =}\StringTok{ "~ -1"}
  
  \ControlFlowTok{if}\NormalTok{(time_vary) \{}
\NormalTok{    time_formula =}\StringTok{ }\KeywordTok{paste0}\NormalTok{(time_formula, }\StringTok{" + "}\NormalTok{, substrate, }\StringTok{" + "}\NormalTok{, enviro)}
\NormalTok{    time_varying =}\StringTok{ }\KeywordTok{as.formula}\NormalTok{(time_formula)}
\NormalTok{    time =}\StringTok{ "year"}
\NormalTok{  \} }\ControlFlowTok{else}\NormalTok{ \{}
\NormalTok{    formula =}\StringTok{ }\KeywordTok{paste0}\NormalTok{(formula, }\StringTok{" + "}\NormalTok{, substrate, }\StringTok{" + "}\NormalTok{, enviro)}
\NormalTok{    time_varying =}\StringTok{ }\OtherTok{NULL}
\NormalTok{    time =}\StringTok{ "year"}
\NormalTok{  \}}
  
  \CommentTok{# fit model. EW commented out quadratic roots, since those are still experimental and won't work for all spp. Also turned}
  \KeywordTok{set.seed}\NormalTok{(}\DecValTok{41389}\NormalTok{) }\CommentTok{# for reproducibility}
\NormalTok{  test_set =}\StringTok{ }\KeywordTok{sample}\NormalTok{(}\DecValTok{1}\OperatorTok{:}\KeywordTok{nrow}\NormalTok{(modeldat), }\DataTypeTok{size =} \KeywordTok{round}\NormalTok{(}\FloatTok{0.1}\OperatorTok{*}\KeywordTok{nrow}\NormalTok{(modeldat)), }\DataTypeTok{replace=}\OtherTok{FALSE}\NormalTok{)}
\NormalTok{  modeldat}\OperatorTok{$}\NormalTok{fold =}\StringTok{ }\DecValTok{1}
\NormalTok{  modeldat}\OperatorTok{$}\NormalTok{fold[test_set] =}\StringTok{ }\DecValTok{2} 
  
  \KeywordTok{print}\NormalTok{(}\StringTok{'running 2-fold CV.'}\NormalTok{)}
  
\NormalTok{  m_cv <-}\StringTok{ }\KeywordTok{try}\NormalTok{( }\KeywordTok{sdmTMB_cv}\NormalTok{( }
    \DataTypeTok{formula =} \KeywordTok{as.formula}\NormalTok{(formula),}
    \DataTypeTok{k_folds=}\DecValTok{2}\NormalTok{,}
    \DataTypeTok{parallel =} \OtherTok{TRUE}\NormalTok{,}
    \DataTypeTok{fold_ids =}\NormalTok{ modeldat}\OperatorTok{$}\NormalTok{fold,}
    \DataTypeTok{time_varying =}\NormalTok{ time_varying,}
    \DataTypeTok{spde =}\NormalTok{ spde,}
    \DataTypeTok{time =}\NormalTok{ time,}
    \DataTypeTok{family =} \KeywordTok{tweedie}\NormalTok{(}\DataTypeTok{link =} \StringTok{"log"}\NormalTok{),}
    \DataTypeTok{data =}\NormalTok{ modeldat,}
    \DataTypeTok{anisotropy =} \OtherTok{FALSE}\NormalTok{,}
    \DataTypeTok{spatial_only =}\NormalTok{ T,}
    \CommentTok{#extra_time argument necessary for prediction?}
    \CommentTok{# extra_time=1980:2100,}
    \DataTypeTok{control=}\KeywordTok{sdmTMBcontrol}\NormalTok{(}\DataTypeTok{map_rf=}\KeywordTok{ifelse}\NormalTok{(spatial_field,F,T))}
\NormalTok{  ),}
  \DataTypeTok{silent=}\NormalTok{T)}
  \ControlFlowTok{if}\NormalTok{(}\KeywordTok{class}\NormalTok{(m_cv)}\OperatorTok{==}\StringTok{'try-error'}\NormalTok{)\{}
    \KeywordTok{print}\NormalTok{(}\KeywordTok{paste}\NormalTok{(}\StringTok{'Error.'}\NormalTok{))}
\NormalTok{  \} }\ControlFlowTok{else}\NormalTok{\{}
    \CommentTok{# tem <- m_cv %>% pluck('data')}
    \CommentTok{# print(paste('data is class',class(tem)))}
\NormalTok{    total_pred_ll =}\StringTok{ }\NormalTok{m_cv }\OperatorTok\StringTok{ }
\StringTok{      }\KeywordTok{pluck}\NormalTok{(}\StringTok{'data'}\NormalTok{) }\OperatorTok\StringTok{ }
\StringTok{      }\NormalTok{dplyr}\OperatorTok{::}\KeywordTok{filter}\NormalTok{(cv_fold}\OperatorTok{==}\DecValTok{2}\NormalTok{) }\OperatorTok\StringTok{ }
\StringTok{      }\KeywordTok{pluck}\NormalTok{(}\StringTok{'cv_loglik'}\NormalTok{) }\OperatorTok\StringTok{ }
\StringTok{      }\KeywordTok{sum}\NormalTok{()}
    \ControlFlowTok{if}\NormalTok{(return_what}\OperatorTok{==}\StringTok{'model'}\NormalTok{) }\KeywordTok{return}\NormalTok{(m_cv)}
    \ControlFlowTok{else} \KeywordTok{return}\NormalTok{(total_pred_ll)}
\NormalTok{  \}}
\NormalTok{\}}
\end{Highlighting}
\end{Shaded}

\begin{Shaded}
\begin{Highlighting}[]
\NormalTok{test_cv <-}\StringTok{ }\KeywordTok{run_sdmTMB_cv}\NormalTok{(}\DataTypeTok{dat=}\NormalTok{trawl_roms_utm,}\DataTypeTok{spp=}\StringTok{'sablefish'}\NormalTok{,}\DataTypeTok{return_what =} \StringTok{'model'}\NormalTok{)}
\end{Highlighting}
\end{Shaded}

\hypertarget{cross-validation-options}{%
\subsection{Cross Validation options}\label{cross-validation-options}}

If we wanted to use cross validation, we could do that in a couple ways.
For example, with a single train/test split, we could assign 10\% of the
observations to the test set and not fit the model for all the folds
(fitting to all folds is the default in sdmTMB\_cv).

\begin{Shaded}
\begin{Highlighting}[]
\KeywordTok{set.seed}\NormalTok{(}\DecValTok{41389}\NormalTok{) }\CommentTok{# for reproducibility}
\NormalTok{test_set =}\StringTok{ }\KeywordTok{sample}\NormalTok{(}\DecValTok{1}\OperatorTok{:}\KeywordTok{nrow}\NormalTok{(modeldat), }\DataTypeTok{size =} \KeywordTok{round}\NormalTok{(}\FloatTok{0.1}\OperatorTok{*}\KeywordTok{nrow}\NormalTok{(modeldat)), }\DataTypeTok{replace=}\OtherTok{FALSE}\NormalTok{)}
\NormalTok{modeldat}\OperatorTok{$}\NormalTok{fold =}\StringTok{ }\DecValTok{1}
\NormalTok{modeldat}\OperatorTok{$}\NormalTok{fold[test_set] =}\StringTok{ }\DecValTok{2}
\KeywordTok{head}\NormalTok{(modeldat}\OperatorTok{$}\NormalTok{fold,}\DecValTok{25}\NormalTok{)}
\end{Highlighting}
\end{Shaded}

Alternatively we could do something like assign all points for a given
year (e.g.~2018) to the test set.

\begin{Shaded}
\begin{Highlighting}[]
\CommentTok{# modeldat$fold = ifelse(modeldat$year=="2018",2,1)}
\end{Highlighting}
\end{Shaded}

A third option is to use blockCV to assign the folds. If you have raster
data, there's a few functions in that package (spatialAutoRange,
rangeExplorer) to estimate the range -- but because those are probably
difficult to estimate with the kind of data we have, I've generally used
ranges in the 50-75km, which is about what's estimated for many WCBTS
species.

\begin{Shaded}
\begin{Highlighting}[]
\NormalTok{the_data <-}\StringTok{ }\NormalTok{sf}\OperatorTok{::}\KeywordTok{st_as_sf}\NormalTok{(modeldat, }\DataTypeTok{coords =} \KeywordTok{c}\NormalTok{(}\StringTok{"longitude"}\NormalTok{, }\StringTok{"latitude"}\NormalTok{),}\DataTypeTok{crs=}\StringTok{"+proj=utm +zone=10 +datum=WGS84 +units=km"}\NormalTok{)}
\NormalTok{sb <-}\StringTok{ }\KeywordTok{spatialBlock}\NormalTok{(}
    \DataTypeTok{speciesData =}\NormalTok{ the_data,}
    \DataTypeTok{species =} \StringTok{"xxxxx"}\NormalTok{,}
    \DataTypeTok{theRange =} \DecValTok{5000}\NormalTok{, }\CommentTok{# range should be in meters}
    \CommentTok{# k = 10,}
    \DataTypeTok{selection =} \StringTok{"systematic"}\NormalTok{,}
    \DataTypeTok{showBlocks =} \OtherTok{FALSE}
\NormalTok{  )}
\NormalTok{modeldat}\OperatorTok{$}\NormalTok{fold =}\StringTok{ }\KeywordTok{ifelse}\NormalTok{(sb}\OperatorTok{$}\NormalTok{fold}\OperatorTok{==}\DecValTok{1}\NormalTok{,}\DecValTok{2}\NormalTok{,}\DecValTok{1}\NormalTok{)}
\end{Highlighting}
\end{Shaded}

And then we can use sdmTMB\_cv to do the estimation for each fold.
Because we did the test-train split, we'll fit the model 2x, but just be
interested in the 1st fit.

\begin{Shaded}
\begin{Highlighting}[]
\NormalTok{m_cv <-}\StringTok{ }\KeywordTok{try}\NormalTok{( }\KeywordTok{sdmTMB_cv}\NormalTok{( }
    \DataTypeTok{formula =} \KeywordTok{as.formula}\NormalTok{(formula),}
    \CommentTok{# time_varying = time_varying,}
    \DataTypeTok{spde =}\NormalTok{ spde,}
    \DataTypeTok{k_folds =} \DecValTok{2}\NormalTok{,}
    \DataTypeTok{fold_ids =} \StringTok{"fold"}\NormalTok{,}
    \DataTypeTok{time =} \StringTok{'year'}\NormalTok{,}
    \DataTypeTok{family =} \KeywordTok{tweedie}\NormalTok{(}\DataTypeTok{link =} \StringTok{"log"}\NormalTok{),}
    \DataTypeTok{data =}\NormalTok{ modeldat,}
    \DataTypeTok{anisotropy =} \OtherTok{FALSE}\NormalTok{,}
    \DataTypeTok{spatial_only =}\NormalTok{ T,}
    \CommentTok{#extra_time argument necessary for prediction?}
    \DataTypeTok{extra_time=}\DecValTok{1980}\OperatorTok{:}\DecValTok{2100}\NormalTok{,}
    \DataTypeTok{map_rf=}\NormalTok{T}
    \CommentTok{# map_rf=ifelse(spatial_field,F,T)}
\NormalTok{  ), }
  \DataTypeTok{silent=}\OtherTok{TRUE}\NormalTok{)}

\CommentTok{# predicted values for the 2nd fold (test)}
\NormalTok{m_cv}\OperatorTok{$}\NormalTok{data}\OperatorTok{$}\NormalTok{cv_predicted[}\KeywordTok{which}\NormalTok{(m_cv}\OperatorTok{$}\NormalTok{data}\OperatorTok{$}\NormalTok{cv_fold}\OperatorTok{==}\DecValTok{2}\NormalTok{)]}
\CommentTok{# log likelihood values for the 2nd fold (test)}
\NormalTok{m_cv}\OperatorTok{$}\NormalTok{data}\OperatorTok{$}\NormalTok{cv_loglik[}\KeywordTok{which}\NormalTok{(m_cv}\OperatorTok{$}\NormalTok{data}\OperatorTok{$}\NormalTok{cv_fold}\OperatorTok{==}\DecValTok{2}\NormalTok{)]}

\NormalTok{total_pred_ll =}\StringTok{ }\KeywordTok{sum}\NormalTok{(m_cv}\OperatorTok{$}\NormalTok{data}\OperatorTok{$}\NormalTok{cv_loglik[}\KeywordTok{which}\NormalTok{(m_cv}\OperatorTok{$}\NormalTok{data}\OperatorTok{$}\NormalTok{cv_fold}\OperatorTok{==}\DecValTok{2}\NormalTok{)])}
\end{Highlighting}
\end{Shaded}

This is the end of this script. Moving the actual modelling (i.e., the
calling of this function) to a new script.

\end{document}
